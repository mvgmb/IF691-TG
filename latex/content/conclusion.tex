\section{Conclusion}

QUIC has proven to be a better alternative to TCP on unreliable networks, addressing multiple problems of TCP when handling packet loss. Another advantage is the incorporation of the TLS protocol, forcing all connections to be encrypted. Relying on UDP and running on user-space, makes it compatible with existing network equipment and can be implemented by any application.

However, the experiments showed that on a cloud environment, where the network is extremely reliable, QUIC performed worse than TCP. Tuning kernel parameters to improve UDP traffic did not bring a significant advantage to QUIC. Additionally, it was also observed that QUIC requires more compute resources when compared to other protocols, increasing the cost of applications that use it.

HTTP/3, which is based on QUIC, suffers from similar problems and showed poor performance for interservice communication. It was not possible to push the protocol to its full potential, as most of its features are better used by a browser rather than in a cloud environment.

QUIC and HTTP/3 are still a great solution for user-facing applications, improving the experience for the end-user with an extra cost for the server. But it doesn’t have a great fit with internal networks, where packet loss is extremely low. In that case it is better to use more traditional protocols that rely on TCP, even with TLS as an additional layer.

The protocol is relatively new and it’s possible that in the future it becomes more competitive in reliable networks. In the meanwhile, TCP/TLS is a great solution that has been supporting the majority of the Internet for years, and it is not going away anytime soon.
